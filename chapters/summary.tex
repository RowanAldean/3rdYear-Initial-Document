\chapter{Summary}
We have looked at the project in adequate detail for stakeholders
to thoroughly understand the system being built. We've made a clear distinction between
future plans for \textit{the product} and the current plans for \textit{the project} (\cref{sec:intro_motivation}).
\par
To more broadly summarise, we'll be using Flutter and Express.js to 
implement the project described in \cref{chap:intro}. We'll
be using a Kanban-style project board with a feature-driven development model
to actively implement the features set out in \cref{sec:features}. The core focus
will be on the jump calculator as it's the unique aspect of the application being developed
and involves substantial difficulty when being created using frameworks for hybrid apps (as seen
by the lack of these frameworks and lack of cross-platform calculators) (\cref{chap:research}).
If time allows, we'll be implementing an ``administrator'' role for the application
as well as other extra features.
\par
This document has described at a high-level the entirety of the project plan,
however it's worth mentioning that a thorough cost-benefit analysis
and feasibility report as well as database and system design (in detail)
would add tremendous value to the development process. Whilst these are not 
explicitly required from this document, they would be near essential in a corporate environment
and would form the basis for the project proposal being accepted/rejected.
\par 
Given enough time (months not weeks) to write a full project plan, we could collate a
fully comprehensive document detailing the technology choices, strategies and deployment methods
for developing a fitness application for training (with a vertical jump calculator).