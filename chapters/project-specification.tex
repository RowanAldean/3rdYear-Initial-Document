\chapter{Project Specification}
\label{chap:project-specification}
\vspace{-10mm}
\section{Feature specification}
\label{sec:features}
The following is a summary of the core features we aim to provide; these will help in 
fulfilling the aims we set in \cref{chap:intro}.
\begin{itemize}[noitemsep]
    \item CRUD (Create, Read, Update, Delete) operations for a user account.
    \item A vertical jump calculator, operating via video upload (\textit{not recording through the app}).
    \begin{itemize}[nolistsep,noitemsep]
        \item Core: Scrub through the video to set the takeoff/landing point.
        \item Extra: Record video on the app (using the phone's camera).
    \end{itemize}
    \item Create a workout.
    \begin{itemize}[nolistsep,noitemsep]
        \item Core: From existing exercises.
        \item Extra: From custom uploaded exercises.
    \end{itemize} 
    \item Subscribe to a preset workout.
    \item A calendar for monitoring progress.
    \item A ``traffic light system'' for monitoring consistency.
    \item A means of communication with other users.
    \begin{itemize}[nolistsep,noitemsep]
        \item Potentially direct messaging or forum style messaging.
    \end{itemize}
    \item Provide video playback of exercises
    \item Allow measurement via metric \& imperial units (primarily for US users).
    \item Provide timer \& stopwatch functionality.
    \item Provide push notification reminders for calendar scheduled workouts.
    \item Provide notes functions for each workout.
    \begin{itemize}[nolistsep,noitemsep]
        \item Extra: Notes per exercise (allowing specific notes).
    \end{itemize}
\end{itemize}
\pagebreak


\section{Requirements}
\label{sec:requirements}
Using the above feature set (\cref{sec:features}), we can place features into functional requirements.
Other additional requirements will make up our non-functional requirements.
\vspace{-3mm}
\subsection{Functional requirements}
\begin{itemize}[noitemsep]
    \item CRUD for a User account.
    \item Upload video from storage.
    \item Scrub through video and set takeoff/landing.
    \item Calculate vertical jump in given units (metric or imperial).
    \item Support video(s) at different frame rate(s).
    \item Track vertical jump progress.
    \item Add/remove a workout to or from a user's calendar.
    \item Mark workouts as complete (with some visual feedback).
    \item Make notes regarding a specific workout.
    \item Provide push notification reminders at set duration(s) before a
    workout.
    \item Video playback from a source (embedded or from our own data storage).
    \item Stopwatch \& timer must work correctly.
    \item Simple chat/forum with other users.
    \item Calendar available on dashboard.
    \item Create a workout from existing exercises.
\end{itemize}
\subsection{Non-functional requirements}
\begin{itemize}[noitemsep]
    \item The app should be cross-platform (run on both iOS and Android).
    \item Have a consistent margin of error (under fixed conditions), when measuring vertical jump.
    \item Visualise workout consistency on user's dashboard.
    \item Operate smoothly and provide a close-to-native experience. Users must not
    feel delayed by the app if/when interviewed following usage.
    \item Aid users workout experience (aside from the vertical jump calculator).
    \item Be reliable (no unexpected data loss).
\end{itemize}
\pagebreak

\section{Technology Stack \& Considerations}
\label{sec:tech-stack}
\vspace{-3mm}
\subsection{Frontend technologies}
\vspace{-3mm}
Given that the app will be cross-platform there are
a few largely accepted technologies for achieving this with a single codebase.
Namely, WebView frameworks (such as Apache Cordova, Ionic, and PhoneGap),
``React Native'' (created by Facebook) and ``Flutter'' (created by Google). 
\par
We'll be using \textbf{Flutter for the frontend} development for multiple reasons.
Commonly, developers opt to use React Native due to its large
community support - however in 2021, Flutter has a very similar ecosystem in my opinion.
Furthermore, whilst React Native is not noticeably slower than Flutter (in many cases) there is a fundamental
difference in how these software frameworks operate (and product native platform code from your high-level codebase).
React Native uses a JavaScript bridge to communicate between native code modules; on the other hand,
Flutter has native component availability by default - in practice the difference in performance is under
a second (milliseconds) but it's a consideration that holds weight as complexity of a project grows.
\par
Flutter is written using Dart (a rarely used language that is predominantly only used for Flutter development).
This is a stark change from JavaScript (used for React Native and more) which I have had previous experience with.
\textit{The project} is an opportunity for me to expand my skill set and learn to work with Dart when using
the Flutter framework - something which I have little experience with at present. Additionally,
JavaScript will be used for \textit{the product} during deployment.
\par
The final reason worth mentioning is Flutter comes with out-of-the-box testing features
as a framework - meaning there is less trouble finding a good workflow.
All of this in addition to the plethora of default UI components and APIs\footnote{Application programming interface(s)} means
that Flutter will be perfect for any form of iterative approach 
to software development.
\par
Avoiding writing native code means we can use 1 single codebase and 
easily satisfies our aims of providing a hybrid mobile application.
Avoiding WebView frameworks is primarily due to their functionality being noticeably
weak and difficult to scale, given that they are rendering HTML, CSS 
and JavaScript in what is essentially a browser window. The application would function,
but not smoothly as stress accumulates when dealing with increasingly large
volumes of workout and application data.

\subsection{Backend technologies}
For the backend data storage, \textbf{we'll be using MongoDB}
- the most popular NoSQL database currently in use.
In large part, this is being used as an opportunity for me
to utilise NoSQL databases as part of a large project as well as
learning to design NoSQL databases from scratch. Ordinarily
developers are typically assigned to projects that have been set-up
by other engineers thus they aren't typically involved in the
architectural stages of system design - including the considerations made when creating database schema.
\par
Personal motivation aside, NoSQL databases are hugely beneficial in a variety of ways.
Firstly, we have the ability to handle large amounts of data at high speeds.
This is achieved using a ``scale-out architecture'' - this simply means that instead
of relying on more computing power (as is the case in traditional ``scale-up architecture'')
we can scale by adding nodes. Related to this is the removal of SQL; not only
are we now avoiding potentially complex and inefficient queries but by
removing the ``SQL magic'' developers can focus on elegant application solutions instead
of spending precious development hours optimizing SQL queries for the database schema that
is in use.
\par
From a data perspective, NoSQL means we can store all kinds of data
structures (known formally as ``models'') instead of the SQL default - tables, rows, columns.
MongoDB supports all forms of raw, semi-structured and structured data; this can be as simple
as binary data, or creating ``documents'' that represent highly structured data such as ``User accounts''
with multiple references.
\par
Overall MongoDB is being used primarily due to it being the most popular NoSQL database and the inherent
community support that coincides with this will be greatly beneficial during development.
It's important to note that poor implementation of NoSQL databases
can mean operating at shockingly slow speeds. Due to the nature of the flexibility and scalability
involved with these databases they are also more susceptible to duplicate data - it is much more difficult
to remove this redundancy than traditional relational models where we can follow a normalization process.
Often traditional normalization makes use of joins, which are not easily executed using MongoDB
and can easily lead to inferior code.
\pagebreak

\textbf{To expose the backend data for the apps to consume 
we'll be using Express.js} - a backend web application
framework for \textbf{Node.js} used in building APIs. Routing is simple and a fully functional
basic API can be written in under 100 lines of code. This level of detail
is not essential at this stage but noting our use of Express will hopefully
help in rationalizing the more high-level decisions outlined above.
\par
Naturally, the technology stack used may include other middle layer
technologies, but it's more than likely we focus our efforts on
Flutter \& MongoDB before making further decisions -
for example, we may introduce a cache layer using Redis to optimize
queries and make our product more efficient. This iterative behavior is natural in 
software engineering and so there's an abundance of changes that can occur in our stack which will ease
development. This doesn't mean radical changes to our simple Flutter-Express-Node stack
but hopefully explains why it's difficult to document any packages and frameworks
we intend to use without a thorough system design stage (often taking months).
\vspace{-5mm}
\subsection{Security and deployment considerations}
The application backend will be hosted on the cloud using
AWS (Amazon Web Services). This is easier to set up than
self-hosting on dedicated servers and avoids the configuration needed to create
a secure system. The costs are predictable based on the usage we see, and
we can easily scale without having to invest in additional server hardware.
AWS is more configurable than other major cloud providers
but provides greater support versus IAC\footnote{Infrastructure-as-code}
tools like Terraform.
\par
We also plan on using fastlane as the CD component of the CI/CD pipeline for deployment.
GitHub Actions can also be used during growth stages before moving to a traditional
CI/CD tool like Jenkins - which can also be integrated with many AWS services. 
\par
A note must be made that many of the above technology decisions are made with HR in mind.
To explain this we must think further into the future; avoiding IAC means we're less likely
to need to employ cloud infrastructure engineers (thus reducing costs) and using popular technologies means
we won't struggle to find talent if needed. These decisions 
are largely dependent on what happens with the product and whether the concept
is proven as commercially viable, but that doesn't dismiss the need to account for factors
such as employee churn as early as possible.
