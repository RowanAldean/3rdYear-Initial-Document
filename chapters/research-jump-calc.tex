\section{Existing vertical jump calculators}
There are few \textit{modern} jump calculators on the market right now. None are combined with existing fitness platforms -
our research shows all of them being standalone mobile or web applications (\labelcref{research:fitness-meter,research:whats-my-vert,research:my-jump}).
\par
There are multiple methods of measuring jump height, often this is important to athletes 
where success is caused/correlated with jump ability i.e. Sprinting \cite{jump-sprint-link}, American football, Basketball, Volleyball.
We also know vertical jump is an important test to assess the explosive strength of the leg
musculature of athletes \cite{aspects-of-strength,nsa-strength-in-athletes}.
We'll be using the time-in-air (TIA) to estimate jump height. Whilst this is inevitably not perfectly
accurate, when using a force plate \footnote{A measuring instrument that measures the ground reaction forces generated by somebody.}
it is no worse than other methods (in terms of consistency) \cite{measuring-jump-paper}.
Thereby still proving useful to the target demographic.
\par
To summarise, the vertical jump calculator created for the project is intended for use as a measurement of progress
and not a concrete result. There will be some deviation from other measurements (such as using a \href{https://www.topendsports.com/testing/products/vertical-jump/vertec.htm}{Vertec}).
\pagebreak 

\subsection{Maths behind calculating vertical jump using video}
\label{research:jump-maths}
Below we'll be considering the physics \cite{hoopsgeek-maths} involved in finding vertical jump height using
\textbf{only smartphone video footage}.
\par
To explain the forces demonstrated in a countermovement jump (CMJ)\footnote{A vertical jump test performed by having an athlete quickly squat to a self-selected depth and then jump as high as possible.}
we'll be considering the 5 phases of action involved (\cref{fig:jump-phases}).
\begin{figure}[H]
	\centering
	\begin{minipage}[c]{0.5\textwidth}
		\caption{Motion phase of countermovement jump (CMJ). 1: Rest. Before jump. \linebreak2: Countermovement. 3: Push to ground. \linebreak4: Jump. Fly. 1 * : Rest. Land. \linebreak Vertical position of torso is y position \cite{jumping-phases-picture}.}
		\label{fig:jump-phases}
	\end{minipage}%
	\begin{minipage}[c]{0.5\textwidth}
		\includegraphics[width=\textwidth]{jump-maths/phases-of-cmj.png}
	\end{minipage}
\end{figure}
\vspace*{-5mm}
\begin{figure}[H]
	\centering
	\resizebox{\linewidth}{!}{
\begin{tikzpicture}
    \centering
    \begin{axis}[
            axis lines = left,
            xlabel = {time in seconds (s)},
            ylabel = {Force in Newtons},
            xmin=-1, xmax=1,
            ymin=0, ymax=2600,
            xtick={-1.0,-0.5,0.0,0.5,1.0},
            ytick={500,1000,1500,2000},
            axis line style={opaque},
            label style={font=\tiny},
            ticklabel style={font=\tiny},
            legend style={draw=none},
        ]

        % Define axis lines
        % Gravity line defined
        \path [
            name path=axis,
        ]
        (axis cs:-1,0) -- (axis cs:1,0);

        
        % Gravity line defined
        \addplot [
            name path=gravity,
            color=gray,
            fill=gray!10!,
            draw=none,
            area legend,
        ]
        {1000};

        %Below the main force of jumper is defined
        \addplot [
            name path=A,
            domain=-1:1,
            samples=11,
            color=red,
            smooth,
            line width=1pt,
            tension={0.15}
        ]
        coordinates {
            (-1,1000)(-0.6,1000)(-0.5,500)(-0.4,1000)(-0.3,1500)(-0.2,2000)(0.0,0.0)(0.5,0.0)(0.6,2100)(0.7,1250)(0.8,1600)(0.9,1000)(1,1000)
        };

        \addplot [
            name path=phase2end,
            domain=-0.6:-0.4,
            samples=11,
            opacity=0,
        ]
        coordinates {
            (-0.4,1000)(-0.2,2000)(0.0,0.0)(0.5,0.0)(0.6,2100)(0.7,1250)(0.8,1600)(0.9,1000)(1,1000)
        };

        \addplot [
            name path=phase3start,
            domain=-0.4:-0.2,
            samples=11,
            opacity=0,
        ]
        coordinates {
            (-0.25,1000)(-0.25,1750)(-0.2,2000)(0.0,0.0)(0.5,0.0)(0.6,2100)(0.7,1250)(0.8,1600)(0.9,1000)(1,1000)
        };

        \addplot [
            name path=phase4start,
            domain=-0.4:-0.2,
            samples=11,
            opacity=0,
        ]
        coordinates {
            (-0.1,1000)(0.0,0.0)(0.5,0.0)(0.6,2100)(0.7,1250)(0.8,1600)(0.9,1000)(1,1000)
        };


        \addplot [
            domain=0:0.5,
            samples=11,
            color=red,
            line width=2pt,
        ]
        coordinates {
            (0,0)(0.5,0)
        };

        % Coloring the phase 1 decline blue
        \addplot [
            fill=blue,
            fill opacity = 0.2,
        ]
        fill between[
            of=A and phase2end
        ];
        % Coloring the phase 1 incline orange
        \addplot [
            fill=orange,
            fill opacity = 0.2,
        ]
        fill between[
            of=phase2end and phase3start
        ];
        % Coloring the phase 3 decline orange
        \addplot [
            fill=yellow,
            fill opacity = 0.2,
        ]
        fill between[
            of=phase3start and phase4start
        ];
        % Coloring final phase
        \addplot [fill=green,fill opacity = 0.2] fill between[
            of=A and gravity, soft clip={domain=0.55:1}
        ];

        % Coloring the gravity line
        \addplot [gray!10!] fill between[
            of=A and axis, soft clip={domain=-1:-0.4}
        ];
        \addplot [gray!10!] fill between[
            of=gravity and axis, soft clip={domain=-0.4:-0.1}
        ];
        \addplot [gray!10!] fill between[
            of=A and axis, soft clip={domain=-0.1:0.55}
        ];
        \addplot [gray!10!] fill between[
            of=gravity and axis, soft clip={domain=0.55:1}
        ];

        \legend{
            \tiny Force of Gravity,
            \tiny Force of Jumper Movement,
        }


        % The phase 1 bending label (70Ns)
        \addplot[mark=none, color = blue] coordinates {(-0.475,930)} node[pin=100:{\tiny 70Ns}]{} ;
        % Phase grey marks
        \addplot[mark=*, color=black!80!, draw opacity = 0] coordinates {(-0.6,1000)(-0.4,1000)(-0.26,1750)(0,0)(0.5,0)};
        % The phase 2 label (70Ns)
        \addplot[mark=none, color = orange] coordinates {(-0.31,1250)} node[pin=100:{\tiny 70Ns}]{} ;
        % The phase 3 label (245Ns)
        \addplot[mark=none, color = yellow] coordinates {(-0.16,1250)} node[pin=80:{\tiny 245Ns}]{} ;
        % The phase 4 label (245Ns)
        \addplot[mark=none, color = green] coordinates {(0.600,1250)} node[pin=100:{\tiny 245Ns}]{} ;

        
    \end{axis}
\end{tikzpicture}}
	\vspace*{-5mm}
	\caption{Ground reaction forces during vertical jump.}
	\label{fig:jump-phases-graph}
\end{figure}
\pagebreak

We'll be using the above, simplified force plate analysis (\cref{fig:jump-phases-graph}),
to delve further into the phases of a jump and the mathematics. 
In reality the force curve wouldn't be so smooth
but this should demonstrate the phases quite clearly.
\par
\textbf{Phase 1: Rest (Pre-jump)}
\par
Before jumping we can see a flat line at $981$ Newton (\cref{fig:jump-phases-graph}). The athlete isn't jumping, but this is gravity doing it's
work. It's easily explained as $F=m*g$, where $m$ is the mass of the athlete and
$g$ is the acceleration of earth's gravity. $F$ is the force needed to counter the effects of gravity. We know that $g=9.81m/s^2$ on earth, thus:
\[
	\displaystyle
	\begin{aligned}
	981N &= m * 9.81 m/s^2 \\
	=> m &= \frac{981N}{9.81 m/s^2} = 100kg
	\end{aligned}	
\]
Here you can see that the force plate is basically acting as a weighing scale, showing
the force gravity exerts on the athlete.
\par
\textbf{Phase 2: Countermovement}
\par
As seen in \cref{fig:jump-phases}, the countermovement involves
the bending of the knees and typically swinging of the arms to lower the center
of gravity before the jump. The force plate (\cref{fig:jump-phases-graph}) is 
registering forces lower than 981N, which means the athlete is accelerating
in a downward movement.
\par
Recall Newton's 3rd Law of motion; 
which states whenever two objects interact, they exert equal and opposite forces on each other.
Knowing the above, we can define Ground Reaction Force (GRF) as ``the force exerted by the ground
on a body in contact with it''. 
We can describe the forces acting at this time as follows:
$$F_{Jumper} = F_{GRF} - F_{Gravity} <=0$$
We can also use the analysis (\cref{fig:jump-phases-graph}) to find
the speed the athlete is moving at before the jump. We know that
$F=ma$ (mass * acceleration), and force is happening over time.
Thus, $Ft=mat$.  
We also know force isn't a constant, it's a function of time and that we can define velocity as $v=at$;
so our formula is:
$$\displaystyle
\int_{t_1}^{t_2} F_{Jumper}(t) \mathrm{d}t = mv$$
where $F_{Jumper}(t)$ is the difference between the GRF and gravity.
\pagebreak
The integral can be calculated from the force plate data. The downward impulse\footnote{Impulse is the integral of a force.} made by
the athlete is shown in the graph (\cref{fig:jump-phases-graph}) as the blue area
below the line representing gravity.
\par
We've assumed the integral (impulse) is -70Ns\footnote{Newton seconds are the units of impulse.}. 
With this, we can conclude that:
\begin{align*}
& \int_{t_1}^{t_2} F(t) \mathrm{d}t = m v \\
&=> -70 N s = 100kg * v \\
&=> v = -70 N s / 100kg = -0.7 m/s
\end{align*}
Therefore, our athlete reaches a peak velocity of -0.7m/s during the countermovement phase.


\subsection{FitnessMeter - Test \& Measure}
\label{research:fitness-meter}
\subsection{What's My Vertical?}
\label{research:whats-my-vert}
\subsection{My Jump 2}
\label{research:my-jump}


\begin{itemize}
	\item \href{https://www.thehoopsgeek.com/measurement-app/#manual}{HoopsGeek App}
	\item \href{https://apps.apple.com/us/app/fitnessmeter-test-measure/id477488986}{Old ass iphone app}
	\item \href{https://apps.apple.com/gb/app/my-jump-2/id1148617550#?platform=iphone}{MyJump2 itunes}
	\item \href{https://www.youtube.com/watch?v=tIBiHDyev6w}{MyJump2 in use}
	\item \href{https://www.topendsports.com/testing/products/vertical-jump/video.htm}{Maths}
	\item \href{https://www.thehoopsgeek.com/the-physics-of-the-vertical-jump/}{Detailed maths courtesy of thehoopsgeek - legend}
\end{itemize}