% !TEX root = ../thesis.tex
\chapter{Introduction}
\raggedright
\label{chap:intro}
This document will outline the plan for development of a functional codebase for a fitness application.
This process, herein referred to as ``the project'' is different to the application(s) produced as a result.
The main difference is that the project is not concerned with the deployment, maintenance and future updates
of the fitness application and no additional academic benefit is gained from creating the
infrastructure or platform-specific functionality. This is to say that the creation
of \textbf{both} a mobile and a web application is not needed for the project but makes up
a core offering of the product for end users following deployment.
\par
\noindent
A \textbf{mobile} application will be produced during the project, which will have 2 main use cases:
\begin{enumerate}
	\item A user will be able to pay (via monthly installments or upfront) for a
	      fitness plan; where they can monitor, make notes and track their
	      progress and routine via a calendar.
	\item A user can create their own fitness plan by producing their own routine using
	      available exercises on the app - the additional functions such as the calendar remain
	      the same.
\end{enumerate}
\pagebreak
\section{Motivations}
\label{sec:intro_motivation}
\hyphenpenalty=100000
The motivation for the project began following an initial consultation with a Californian
NBA-level basketball athletic trainer who was looking for a bespoke application
to reduce his business expenses. Initially, 4 days of planning and a 40-minute discussion
took place before the business decided that the costs associated with building this platform
are not a priority for the time being.
\par
In the process of considering the development of said fitness platform, I spoke to developers
working with smaller high-level sports trainers regarding the technology used and their clients
future plans. At this time fitness influencers Krissy Cela and Jack Bullimore had proven the concept
a year earlier with the rollout of ``\href{https://toneandsculpt.app/}{Tone and Sculpt}'' -
grossing over GBP 1 million in the 8 months following its January 2019 release \cite{tonensculpt}.
My initial desire to pursue a project in this space grew more after hearing the story and approach taken by
``Tone and Sculpt'' on the Diary of a CEO podcast \cite{krissy-podcast}.
\par
In the world of athletic sports there is currently no viable, consistent and widely used means
of measuring vertical leap besides
\href{https://www.topendsports.com/testing/products/vertical-jump/vertec.htm}{the Vertec}.
Prices on these devices can be as high as £500 and cheaper alternatives commonly found
at a youth/amateur level - that involve markings on walls - don't allow for an approach
jump (without the risk of hitting the wall). This is being mentioned because a large motivation for the jump calculator
functionality is to allow an easy and accessible means of measuring a users vertical leap
progress. Naturally, the use of a smartphone for measurement is not completely accurate but
will provide sufficient consistency (given the user's setup) and this can be used to monitor
improvement.
\par
From a trainers' perspective,  using PDFs to release fitness plans is cumbersome
and involves finding a suitable workflow and rollout. Further to the fact that they're
easily copied and distributed. Existing white-label\footnote{A white-label product is a product or service produced by one company that other companies rebrand to make it appear as if they had made it.}
 mobile app solutions to these issues
don't offer much customization beyond simple re-branding; many don't support video.
Following discussion with fitness trainers in various countries (including the USA, UAE and UK)
I concluded that this is something that needs to be addressed. Many robo-fitness apps
exist for end users, but few include real trainers and even less give
freedom to the administrator.
\par
In summary, the motivation for the project is to give a customizable fitness planning
solution to those interested in monitoring their training. This solution should be
usable by administrators (trainers/teams), average ``gym go-ers'' and
highly trained athletes alike. The global fitness app market size
was valued at USD 4.4 billion in 2020 and is expected to expand at a
compound annual growth rate (CAGR) of 21.6\% from 2021 to 2028 \cite{fitness-app-market-size}.
We can see that the total addressable market is huge and set to grow quickly,
if this is any indicator of the volume of potential users looking for a fitness solution then the proposed
project should prove tremendously useful at solving the aforementioned problems.
\pagebreak

\section{Overview \& Aims}
\label{sec:intro_overview}
The projects aims include the following - some of these aims will be broken down into
succinct functional and non-functional requirements in \cref{sec:requirements}:
\begin{itemize}
	\item The ability for a user to plan their workouts on a calendar.
	\item The ability for a user to track the progress of a workout.
	\item The ability for a user to measure their vertical leap using their phone.
	\item Allow users to keep track of their progress in relevant areas
	      such as weight, strength measurements etc. more efficiently than traditional methods.
	\item Allow trainers to comfortably deliver fitness plans to
	      clients without a steep technical learning curve.
\end{itemize}
The core overview of the project is that we're looking to develop a mobile application that
allows a user to register an account before having the ability to follow/create a fitness plan
and provide relevant tools for consistency and the successful reach of a users goals
(irrespective of their background). The project aims to focus on the 
end user\footnote{The user training using the app} however considerations will be made for 
the complete implementation of an administrator role (time permitting).
\par
The project is concerned with the ability to solve both parties problems through the development
of the application, but the primary focus is on the development of the core functionality
and the vertical jump calculator. From a business perspective we can consider the jump calculator
the unique selling point of \textit{the product} and
a complex component of \textit{the project}.
