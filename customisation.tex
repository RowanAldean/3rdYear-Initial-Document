% !TEX root = thesis.tex
%% Here you can specify new packages, commands and environments that you intend
%% to use. Using custom commands (for example, those for e.g., i.e., etc. below)
%% can make your document easier to write, read and more consistent.
\usepackage{float} % Adds the [H] option for forced figure placement
\usepackage[norefs,nocites]{refcheck} % Check and warn about unused labels
\usepackage[export]{adjustbox} % Used to add a frame around figures

% We are writing in British English
\usepackage[british]{babel}

% For formatting mathematical equations
\usepackage{amsmath}
% For plotting graphs
\usepackage{pgfplots}
\usepgfplotslibrary{fillbetween}
% Captions on the side


% How many levels of sections/subsections etc to display in the Table of Contents
\setcounter{tocdepth}{1}

% Space out lines slightly - 1.5 spacing is overly severe, so we go for a 
% more visually pleasing 1.31
\linespread{1.31}

% Paragraph spacing
\setlength{\parskip}{0.5em}

% Common shortcuts for consistency - this allows you to write, for example, 
% \eg in LaTeX instead of typing e.g., so that every single instance will be 
% formatted identically. If you later want to change one of these definitions, 
% all usages throughout the document will be updated.
\newcommand{\eg}{e.g.,\xspace}
\newcommand{\ie}{i.e.,\xspace}
\newcommand{\etc}{etc.\@\xspace}
\newcommand{\cf}{cf.\xspace}
\newcommand{\vs}{vs.\xspace}
\newcommand{\etal}{et al.\xspace}
\newcommand{\sd}{s.d.\xspace}
\newcommand{\elide}{[\,\ldots]\xspace}
\newcommand{\edots}{\,\ldots}

% Figure caption formatting
\usepackage[font=small,skip=1em, justification=raggedright]{caption}
% \usepackage[labelformat=simple]{subfig}
% \renewcommand{\thesubfigure}{\alph{subfigure})}
\usepackage[]{subcaption}

% Allow multiple columns
\usepackage{multicol}
\usepackage{enumitem}

% Put footnotes at bottom of page always
\usepackage[bottom]{footmisc}

% Code listing formatting
\usepackage[final]{listings} % "final" option means show listings even in draft mode
\usepackage{lstautogobble}
\usepackage{sourcecodepro}
\pdfmapfile{=SourceCodePro.map}
\lstset{
	xleftmargin=0.5cm,frame=tlbr,framesep=4pt,framerule=0.5pt,
	language=,
	upquote=true,
	columns=fixed,
	tabsize=2,
	extendedchars=true,
	breaklines=true,
	numbers=left,
	numbersep=10pt,
	basicstyle=\ttfamily\scriptsize,
	numberstyle=\tiny,
	stringstyle=\ttfamily,
	captionpos=b,
	showstringspaces=false,
	autogobble=true
}



% Use IEEEtran citation style
\bibliographystyle{IEEEtran} 